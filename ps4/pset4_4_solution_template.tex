%
% 6.006 problem set 4 solution template
% You can ignore the first 50 lines here.  Search for TODO.
%
\documentclass[12pt,oneside]{article}

\usepackage{amsmath}
\usepackage{hyperref}
\usepackage{color}
\usepackage{enumerate}
\usepackage{textcomp}
\usepackage[T1]{fontenc}
\usepackage{clrscode}

\def\MLine#1{\par\hspace*{-\leftmargin}\parbox{\textwidth}{\[#1\]}}

\newcommand{\name}{}

\newcommand{\mitst}[1]{\begin{description}
\item[MIT students:] #1
\end{description}}
\newcommand{\smast}[1]{\begin{description}
\item[SMA students:] #1
\end{description}}

\newcommand{\profs}{Erik Demaine, Ronald L. Rivest, and Nir Shavit}
\newcommand{\subj}{6.006}

\newlength{\toppush}
\setlength{\toppush}{2\headheight}
\addtolength{\toppush}{\headsep}

\newcommand{\htitle}[3]{\noindent\vspace*{-\toppush}\newline\parbox{6.5in}
{\textit{Introduction to Algorithms: 6.006}\hfill\name\newline
Massachusetts Institute of Technology \hfill #3\newline
\profs\hfill Handout #1\vspace*{-.5ex}\newline
\mbox{}\hrulefill\mbox{}}\vspace*{1ex}\mbox{}\newline
\begin{center}{\Large\bf #2}\end{center}}

\newcommand{\handout}[3]{\thispagestyle{empty}
 \markboth{Handout #1: #2 \hfill \yourname}{Handout #1: #2 \hfill \yourname}
 \pagestyle{myheadings}\htitle{#1}{#2}{#3}\pagenumbering{gobble}}

 
\newcounter{problemnum}
\newcommand{\theproblem}{Problem \theproblemsetnum-\theproblemnum}
\newenvironment{problems}{
        \begin{list}{{\bf \theproblem. \hspace*{0.5em}}}
        {\setlength{\leftmargin}{0em}
         \setlength{\rightmargin}{0em}
         \setlength{\labelwidth}{0em}
         \setlength{\labelsep}{0em}
         \usecounter{problemnum}}}{\end{list}}
\makeatletter
\newcommand{\problem}[1][{}]{\item \let\@currentlabel=\theproblem \textbf{#1}}
\makeatother

\newcounter{problempartnum}[problemnum]
\newenvironment{problemparts}{
        \begin{list}{{\bf (\alph{problempartnum})}}
        {\setlength{\leftmargin}{2.5em}
         \setlength{\rightmargin}{2.5em}
         \setlength{\labelsep}{0.5em}}}{\end{list}}
\newcommand{\problempart}{\addtocounter{problempartnum}{1}\item} 
 
\newcommand{\points}[1]{[#1 points]\ }
\newcommand{\parts}[1]
{
  \ifnum#1=1
  (1 part)
  \else
  (#1 parts)
  \fi
  \ 
}
 
 
\setlength{\oddsidemargin}{0pt}
\setlength{\evensidemargin}{0pt}
\setlength{\textwidth}{6.5in}
\setlength{\topmargin}{0in}
\setlength{\textheight}{8.5in}



% Fill these in!
\newcommand{\theproblemsetnum}{4}
\renewcommand{\theproblemnum}{4}
\newcommand{\releasedate}{October 22, 2013}
\newcommand{\tabUnit}{3ex}
\newcommand{\tab}{\hspace*{\tabUnit}}

\newcommand{\yourname}{YOUR NAME HERE}  %TODO
\newcommand{\yourcollaborators}{YOUR COLLABORATORS HERE} %TODO

\begin{document}

\begin{center}
\handout{\# 4}{Problem Set \theproblemsetnum, Problem \theproblemnum}{\releasedate}
\begin{large} \yourname \end{large} \\
\begin{large} Collaborators:  \yourcollaborators \end{large} \\
\end{center}

\hrulefill
\medskip

This problem set is due {\bf Tuesday, November 5} at {\bf 11:59PM}. \\

This solution template should be turned in through \color{blue} \href{https://alg.csail.mit.edu}{our submission site}. \color{black}   \footnote{Register an account, if you haven't done so.  Then go to Homework, Problem Set 4, and upload your files.  } \\

For written questions, full credit will be given only to
correct solutions that are described clearly {\em and concisely}. \\

\medskip

\hrulefill

\newpage

\pagestyle{myheadings}


\begin{problems}

\problem \points{25} \textbf{Demolition Derby}

Tonight is the night of the Big Demolition Derby! It will be starring
some of the baddest, toughest cars around duking it out in an $N$ by
$N$ arena. At any given time, each space of the arena can either be
empty, have a single car in it, or have a boulder in it. At each time
step, only one car is allowed to move and they can only move one space
in either of the vertical or horizontal directions, but they cannot
move into a space with a boulder in it. If a car moves into a space
where there previously existed a car, the sitting car explodes into
dust.

Thus, given the size of the arena $N$, the list of boulder locations
$B$, and the list of car locations $C$ we want to find an ordered list
of car moves that will result in there being only one car left using the
fewest number of moves. Each move should be represented as a tuple of
tuples in the form $((\id{fromX},\id{fromY}), (\id{toX},\id{toY}))$.

For example, the following input:
%
\begin{eqnarray*}
  N &=& 4 \\
  B &=& ( (1,1), (2,2) ) \\
  C &=& ((0,0), (0,3), (2,1), (3,2))
\end{eqnarray*}
%
 would refer to the following diagram where the B's represent boulders and the C's represent cars and the \texttt{-}'s represent empty spaces. Note that $x$ coordinates increase from left to right and $y$ coordinates increase from top to bottom.
\begin{verbatim}
    C  -  -  -
    -  B  C  -
    -  -  B  C
    C  -  -  -
\end{verbatim}
The fewest number of moves that this example can be ``solved'' in is
$8$ moves. One example of such an $8$ move solution is:
\begin{verbatim}
 [((0,3), (0,2)), ((3,2), (3,1)), ((3,1), (2,1)), ((0,2), (0,1)),
  ((0,1), (0,0)), ((0,0), (1,0)), ((1,0), (2,0)), ((2,0), (2,1))]
\end{verbatim}

\begin{problemparts}
\problempart \points{2}
What are the possible states the derby can be in? How might you represent the states in a Python program?

\fbox{\vbox{
%TODO:
Write something here!
}}

\problempart \points {2}
Given a certain state of the derby, what states are just one move away?

\fbox{\vbox{
%TODO:
Write something here!
}}

\problempart \points {2}
What is characteristic of the states that the derby ends at?

\fbox{\vbox{
%TODO:
Write something here!
}}

\problempart \points {2}
Suppose you are given this input:
%
\begin{eqnarray*}
  N &=& 4 \\
  B &=& ( ) \\
  C &=& ((0,0), (0,3), (1,2))
\end{eqnarray*}
%
which may be diagrammed as follows:
\begin{verbatim}
    C  -  -  -
    -  -  -  -
    -  C  -  -
    C  -  -  -
\end{verbatim}
Provide a $4$ move solution.

\fbox{\vbox{
%TODO:
Write something here!
}}

\problempart \points {2}
Explain how a version of breadth first search would be useful in solving this problem.

\fbox{\vbox{
%TODO:
Write something here!
}}

\problempart \points{15}
Implement  {\tt DemoDerby(N, B, C)} in  {\tt pset4\_4f\_solution\_template.py}.

This problem should be submitted using {\tt pset4\_4f\_solution\_template.py}.

\end{problemparts}


\end{problems}

\end{document}
