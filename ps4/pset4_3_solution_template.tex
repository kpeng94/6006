%
% 6.006 problem set 4 solution template
% You can ignore the first 50 lines here.  Search for TODO.
%
\documentclass[12pt,oneside]{article}

\usepackage{amsmath}
\usepackage{hyperref}
\usepackage{color}
\usepackage{enumerate}
\usepackage{textcomp}
\usepackage[T1]{fontenc}
\usepackage{clrscode}

\def\MLine#1{\par\hspace*{-\leftmargin}\parbox{\textwidth}{\[#1\]}}

\newcommand{\name}{}

\newcommand{\mitst}[1]{\begin{description}
\item[MIT students:] #1
\end{description}}
\newcommand{\smast}[1]{\begin{description}
\item[SMA students:] #1
\end{description}}

\newcommand{\profs}{Erik Demaine, Ronald L. Rivest, and Nir Shavit}
\newcommand{\subj}{6.006}

\newlength{\toppush}
\setlength{\toppush}{2\headheight}
\addtolength{\toppush}{\headsep}

\newcommand{\htitle}[3]{\noindent\vspace*{-\toppush}\newline\parbox{6.5in}
{\textit{Introduction to Algorithms: 6.006}\hfill\name\newline
Massachusetts Institute of Technology \hfill #3\newline
\profs\hfill Handout #1\vspace*{-.5ex}\newline
\mbox{}\hrulefill\mbox{}}\vspace*{1ex}\mbox{}\newline
\begin{center}{\Large\bf #2}\end{center}}

\newcommand{\handout}[3]{\thispagestyle{empty}
 \markboth{Handout #1: #2 \hfill \yourname}{Handout #1: #2 \hfill \yourname}
 \pagestyle{myheadings}\htitle{#1}{#2}{#3}\pagenumbering{gobble}}


\newcounter{problemnum}
\newcommand{\theproblem}{Problem \theproblemsetnum-\theproblemnum}
\newenvironment{problems}{
        \begin{list}{{\bf \theproblem. \hspace*{0.5em}}}
        {\setlength{\leftmargin}{0em}
         \setlength{\rightmargin}{0em}
         \setlength{\labelwidth}{0em}
         \setlength{\labelsep}{0em}
         \usecounter{problemnum}}}{\end{list}}
\makeatletter
\newcommand{\problem}[1][{}]{\item \let\@currentlabel=\theproblem \textbf{#1}}
\makeatother

\newcounter{problempartnum}[problemnum]
\newenvironment{problemparts}{
        \begin{list}{{\bf (\alph{problempartnum})}}
        {\setlength{\leftmargin}{2.5em}
         \setlength{\rightmargin}{2.5em}
         \setlength{\labelsep}{0.5em}}}{\end{list}}
\newcommand{\problempart}{\addtocounter{problempartnum}{1}\item}

\newcommand{\points}[1]{[#1 points]\ }
\newcommand{\parts}[1]
{
  \ifnum#1=1
  (1 part)
  \else
  (#1 parts)
  \fi
  \
}


\setlength{\oddsidemargin}{0pt}
\setlength{\evensidemargin}{0pt}
\setlength{\textwidth}{6.5in}
\setlength{\topmargin}{0in}
\setlength{\textheight}{8.5in}



% Fill these in!
\newcommand{\theproblemsetnum}{4}
\renewcommand{\theproblemnum}{3}
\newcommand{\releasedate}{October 22, 2013}
\newcommand{\tabUnit}{3ex}
\newcommand{\tab}{\hspace*{\tabUnit}}

\newcommand{\yourname}{Kevin Peng}
\newcommand{\yourcollaborators}{Genghis Chau, Jodie Chen}

\begin{document}

\begin{center}
\handout{\# 4}{Problem Set \theproblemsetnum, Problem \theproblemnum}{\releasedate}
\begin{large} \yourname \end{large} \\
\begin{large} Collaborators:  \yourcollaborators \end{large} \\
\end{center}

\hrulefill
\medskip

This problem set is due {\bf Tuesday, November 5} at {\bf 11:59PM}. \\

This solution template should be turned in through \color{blue} \href{https://alg.csail.mit.edu}{our submission site}. \color{black}   \footnote{Register an account, if you haven't done so.  Then go to Homework, Problem Set 4, and upload your files.  } \\

For written questions, full credit will be given only to
correct solutions that are described clearly {\em and concisely}. \\

\medskip

\hrulefill

\newpage

\pagestyle{myheadings}


\begin{problems}

\problem \points{25} \textbf{Joan of Arcsin}

\begin{problemparts}

\problempart \points{10}
Explain how you would implement the arcsin function using Newton's
Method, and the given library. It must work correctly and efficiently for all $y$ in the
given range, including the endpoints. You may assume that the given library is correct
and of great accuracy.

\fbox{\vbox{
We start with the equation $x = \arcsin(y)$. Since we can only use mathematical functions in our library with $\sin, \cos,$ and $\tan$, we change this equation to $y = \sin(x)$. Then we subtract from both sides to get $\sin(x) - y = 0$. Thus, we let $f(x) = \sin(x) - y$ and find that $f'(x) = \cos(x)$. We can use the tangent line approximation to get $x_{i+1} = x_i - \frac{f(x_{i})}{f'(x_{i})}$.
}}

\problempart \points{10}
 Explain why one iteration of Newton's Method definitely improves your
approximation to the correct answer (assuming that it is not already exactly the correct
answer).

\fbox{\vbox{
}}

\problempart \points{5}
Explain why your iterative procedure doesn't take you outside the range $[0,\frac{\pi}{2}]$, inclusive.

\fbox{\vbox{
%TODO:
Write something here!
}}

\end{problemparts}


\end{problems}

\end{document}
