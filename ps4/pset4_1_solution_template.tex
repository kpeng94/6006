%
% 6.006 problem set 4 solution template
% You can ignore the first 50 lines here.  Search for TODO.
%
\documentclass[12pt,oneside]{article}

\usepackage{amsmath}
\usepackage{hyperref}
\usepackage{color}
\usepackage{enumerate}
\usepackage{textcomp}
\usepackage[T1]{fontenc}
\usepackage{clrscode}

\def\MLine#1{\par\hspace*{-\leftmargin}\parbox{\textwidth}{\[#1\]}}

\newcommand{\name}{}

\newcommand{\mitst}[1]{\begin{description}
\item[MIT students:] #1
\end{description}}
\newcommand{\smast}[1]{\begin{description}
\item[SMA students:] #1
\end{description}}

\newcommand{\profs}{Erik Demaine, Ronald L. Rivest, and Nir Shavit}
\newcommand{\subj}{6.006}

\newlength{\toppush}
\setlength{\toppush}{2\headheight}
\addtolength{\toppush}{\headsep}

\newcommand{\htitle}[3]{\noindent\vspace*{-\toppush}\newline\parbox{6.5in}
{\textit{Introduction to Algorithms: 6.006}\hfill\name\newline
Massachusetts Institute of Technology \hfill #3\newline
\profs\hfill Handout #1\vspace*{-.5ex}\newline
\mbox{}\hrulefill\mbox{}}\vspace*{1ex}\mbox{}\newline
\begin{center}{\Large\bf #2}\end{center}}

\newcommand{\handout}[3]{\thispagestyle{empty}
 \markboth{Handout #1: #2 \hfill \yourname}{Handout #1: #2 \hfill \yourname}
 \pagestyle{myheadings}\htitle{#1}{#2}{#3}\pagenumbering{gobble}}

 
\newcounter{problemnum}
\newcommand{\theproblem}{Problem \theproblemsetnum-\theproblemnum}
\newenvironment{problems}{
        \begin{list}{{\bf \theproblem. \hspace*{0.5em}}}
        {\setlength{\leftmargin}{0em}
         \setlength{\rightmargin}{0em}
         \setlength{\labelwidth}{0em}
         \setlength{\labelsep}{0em}
         \usecounter{problemnum}}}{\end{list}}
\makeatletter
\newcommand{\problem}[1][{}]{\item \let\@currentlabel=\theproblem \textbf{#1}}
\makeatother

\newcounter{problempartnum}[problemnum]
\newenvironment{problemparts}{
        \begin{list}{{\bf (\alph{problempartnum})}}
        {\setlength{\leftmargin}{2.5em}
         \setlength{\rightmargin}{2.5em}
         \setlength{\labelsep}{0.5em}}}{\end{list}}
\newcommand{\problempart}{\addtocounter{problempartnum}{1}\item} 
 
\newcommand{\points}[1]{[#1 points]\ }
\newcommand{\parts}[1]
{
  \ifnum#1=1
  (1 part)
  \else
  (#1 parts)
  \fi
  \ 
}
 
 
\setlength{\oddsidemargin}{0pt}
\setlength{\evensidemargin}{0pt}
\setlength{\textwidth}{6.5in}
\setlength{\topmargin}{0in}
\setlength{\textheight}{8.5in}

\newcommand{\defn}[1]{{\boldmath\textit{\textbf{#1}}}}
\newcommand{\defi}[1]{{\textit{\textbf{#1\/}}}}


% Fill these in!
\newcommand{\theproblemsetnum}{4}
\renewcommand{\theproblemnum}{1}
\newcommand{\releasedate}{October 22, 2013}
\newcommand{\tabUnit}{3ex}
\newcommand{\tab}{\hspace*{\tabUnit}}

\newcommand{\yourname}{YOUR NAME HERE}  %TODO
\newcommand{\yourcollaborators}{YOUR COLLABORATORS HERE} %TODO

\begin{document}

\begin{center}
\handout{\# 1}{Problem Set \theproblemsetnum, Problem \theproblemnum}{\releasedate}
\begin{large} \yourname \end{large} \\
\begin{large} Collaborators:  \yourcollaborators \end{large} \\
\end{center}

\hrulefill
\medskip

This problem set is due {\bf Tuesday, November 5} at {\bf 11:59PM}. \\

This solution template should be turned in through \color{blue} \href{https://alg.csail.mit.edu}{our submission site}. \color{black}   \footnote{Register an account, if you haven't done so.  Then go to Homework, Problem Set 4, and upload your files.  } \\

For written questions, full credit will be given only to
correct solutions that are described clearly {\em and concisely}. \\

\medskip

\hrulefill

\newpage

\pagestyle{myheadings}


\begin{problems}

\problem \points{25} \textbf{Use Your Tires}

Ben Bitdiddle is trying to travel from Cambridge to Pasadena, CA to
attend a cousin's wedding. He is given an undirected graph
representing a road map to guide his way. Each edge in this graph
represents a road segment, and each vertex represents an
intersection. For this problem, all road edges have lengths $1$, $2$,
or $3$. He wants to minimize the total distance traveled.

But there's a catch. Not all roads are the best to travel on. There
are \defn{bad} roads that are so rough that Ben is guaranteed to lose
exactly one of his tires whenever he travels on a bad road.
He can't fix tires en route so once a tire is gone, it is gone. Ben
can keep traveling as long as he has at least one good tire (his car is one he designed and built himself as part of his Mech. Engr.
double major).

How does Ben find the minimum-length path from Cambridge to Pasadena
that loses at most $3$ tires? Be sure to describe your algorithm as
well as analyze its runtime. Your algorithm should return the path
that Ben would take.


\fbox{\vbox{
%TODO:
Write something here!
}}


\end{problems}

\end{document}
