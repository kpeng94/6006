%
% 6.006 problem set 4 solution template
% You can ignore the first 50 lines here.  Search for TODO.
%
\documentclass[12pt,oneside]{article}

\usepackage{amsmath}
\usepackage{hyperref}
\usepackage{color}
\usepackage{enumerate}
\usepackage{textcomp}
\usepackage[T1]{fontenc}
\usepackage{clrscode}

\def\MLine#1{\par\hspace*{-\leftmargin}\parbox{\textwidth}{\[#1\]}}

\newcommand{\name}{}

\newcommand{\mitst}[1]{\begin{description}
\item[MIT students:] #1
\end{description}}
\newcommand{\smast}[1]{\begin{description}
\item[SMA students:] #1
\end{description}}

\newcommand{\profs}{Erik Demaine, Ronald L. Rivest, and Nir Shavit}
\newcommand{\subj}{6.006}

\newlength{\toppush}
\setlength{\toppush}{2\headheight}
\addtolength{\toppush}{\headsep}

\newcommand{\htitle}[3]{\noindent\vspace*{-\toppush}\newline\parbox{6.5in}
{\textit{Introduction to Algorithms: 6.006}\hfill\name\newline
Massachusetts Institute of Technology \hfill #3\newline
\profs\hfill Handout #1\vspace*{-.5ex}\newline
\mbox{}\hrulefill\mbox{}}\vspace*{1ex}\mbox{}\newline
\begin{center}{\Large\bf #2}\end{center}}

\newcommand{\handout}[3]{\thispagestyle{empty}
 \markboth{Handout #1: #2 \hfill \yourname}{Handout #1: #2 \hfill \yourname}
 \pagestyle{myheadings}\htitle{#1}{#2}{#3}\pagenumbering{gobble}}


\newcounter{problemnum}
\newcommand{\theproblem}{Problem \theproblemsetnum-\theproblemnum}
\newenvironment{problems}{
        \begin{list}{{\bf \theproblem. \hspace*{0.5em}}}
        {\setlength{\leftmargin}{0em}
         \setlength{\rightmargin}{0em}
         \setlength{\labelwidth}{0em}
         \setlength{\labelsep}{0em}
         \usecounter{problemnum}}}{\end{list}}
\makeatletter
\newcommand{\problem}[1][{}]{\item \let\@currentlabel=\theproblem \textbf{#1}}
\makeatother

\newcounter{problempartnum}[problemnum]
\newenvironment{problemparts}{
        \begin{list}{{\bf (\alph{problempartnum})}}
        {\setlength{\leftmargin}{2.5em}
         \setlength{\rightmargin}{2.5em}
         \setlength{\labelsep}{0.5em}}}{\end{list}}
\newcommand{\problempart}{\addtocounter{problempartnum}{1}\item}

\newcommand{\points}[1]{[#1 points]\ }
\newcommand{\parts}[1]
{
  \ifnum#1=1
  (1 part)
  \else
  (#1 parts)
  \fi
  \
}

\newcommand{\defn}[1]{{\boldmath\textit{\textbf{#1}}}}
\newcommand{\defi}[1]{{\textit{\textbf{#1\/}}}}

\setlength{\oddsidemargin}{0pt}
\setlength{\evensidemargin}{0pt}
\setlength{\textwidth}{6.5in}
\setlength{\topmargin}{0in}
\setlength{\textheight}{8.5in}



% Fill these in!
\newcommand{\theproblemsetnum}{4}
\renewcommand{\theproblemnum}{2}
\newcommand{\releasedate}{October 22, 2013}
\newcommand{\tabUnit}{3ex}
\newcommand{\tab}{\hspace*{\tabUnit}}

\newcommand{\yourname}{Kevin Peng}
\newcommand{\yourcollaborators}{None}

\begin{document}

\begin{center}
\handout{\# 2}{Problem Set \theproblemsetnum, Problem \theproblemnum}{\releasedate}
\begin{large} \yourname \end{large} \\
\begin{large} Collaborators:  \yourcollaborators \end{large} \\
\end{center}

\hrulefill
\medskip

This problem set is due {\bf Tuesday, November 5} at {\bf 11:59PM}. \\

This solution template should be turned in through \color{blue} \href{https://alg.csail.mit.edu}{our submission site}. \color{black}   \footnote{Register an account, if you haven't done so.  Then go to Homework, Problem Set 4, and upload your files.  } \\

For written questions, full credit will be given only to
correct solutions that are described clearly {\em and concisely}. \\

\medskip

\hrulefill

\newpage

\pagestyle{myheadings}


\begin{problems}

\problem \points{25} \textbf{Two-colorability}

You are given an undirected graph $G=(V, E)$ which is not necessarily
connected. The graph is \defn{two-colorable} if it is possible to
color each vertex red or blue, such that no two vertices of the same
color are connected by an edge.

Your algorithms in parts b) and c) should run in $O(V+E)$ time. Be sure to describe your
algorithms and analyze its runtime.

\begin{problemparts}
\problempart \points{5}
Argue the following two statements:

\begin{itemize}
\item If there exists an odd-length cycle in the input graph, then the graph is not two-colorable.
\item If there does not exist an odd-length cycle in the input graph, then the graph is two-colorable.
\end{itemize}

\fbox{\vbox{
If there exists an odd-length cycle, we will isolate the cycle and try to color it with only two colors. Without loss of generality, we color some node on the graph a color $c_1$. The next node in the cycle must be color $c_2$. The following nodes must be colors $c_1, c_2, c_1, c_2 \dots$. The last node must be $c_1$. However, this contradicts the two-colorable property, so this cycle is not two-colorable. Thus, the graph is not two-colorable if it contains an odd-length cycle.

If there does not exist an odd-length cycle
}}

\problempart \points{10}
Show how to use DFS to determine whether or not a given undirected input graph $G$ is two-colorable. If the graph is not two-colorable, your algorithm should return an odd-length cycle. (Hint: What happens if you encounter a back edge in your DFS search?)

\fbox{\vbox{
%TODO:
Write something here!
}}

\problempart \points{10}
Show how to use BFS to determine whether or not a given undirected input graph $G$ is two-colorable. If the graph is not two-colorable, your algorithm should return an odd-length cycle. (Hint: What happens if you encounter a cross edge in your BFS search?)

\fbox{\vbox{
%TODO:
Write something here!
}}

\end{problemparts}


\end{problems}

\end{document}
