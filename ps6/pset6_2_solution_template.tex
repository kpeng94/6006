%
% 6.006 problem set 6 solution template
% You can ignore the first 50 lines here.  Search for TODO.
%
\documentclass[12pt,oneside]{article}

\usepackage{amsmath}
\usepackage{hyperref}
\usepackage{color}
\usepackage{enumerate}
\usepackage{textcomp}
\usepackage[T1]{fontenc}
\usepackage{clrscode}

\def\MLine#1{\par\hspace*{-\leftmargin}\parbox{\textwidth}{\[#1\]}}

\newcommand{\name}{}

\newcommand{\mitst}[1]{\begin{description}
\item[MIT students:] #1
\end{description}}
\newcommand{\smast}[1]{\begin{description}
\item[SMA students:] #1
\end{description}}

\newcommand{\profs}{Erik Demaine, Ronald L. Rivest, and Nir Shavit}
\newcommand{\subj}{6.006}

\newlength{\toppush}
\setlength{\toppush}{2\headheight}
\addtolength{\toppush}{\headsep}

\newcommand{\htitle}[3]{\noindent\vspace*{-\toppush}\newline\parbox{6.5in}
{\textit{Introduction to Algorithms: 6.006}\hfill\name\newline
Massachusetts Institute of Technology \hfill #3\newline
\profs\hfill Handout #1\vspace*{-.5ex}\newline
\mbox{}\hrulefill\mbox{}}\vspace*{1ex}\mbox{}\newline
\begin{center}{\Large\bf #2}\end{center}}

\newcommand{\handout}[3]{\thispagestyle{empty}
 \markboth{Handout #1: #2 \hfill \yourname}{Handout #1: #2 \hfill \yourname}
 \pagestyle{myheadings}\htitle{#1}{#2}{#3}\pagenumbering{gobble}}

 
\newcounter{problemnum}
\newcommand{\theproblem}{Problem \theproblemsetnum-\theproblemnum}
\newenvironment{problems}{
        \begin{list}{{\bf \theproblem. \hspace*{0.5em}}}
        {\setlength{\leftmargin}{0em}
         \setlength{\rightmargin}{0em}
         \setlength{\labelwidth}{0em}
         \setlength{\labelsep}{0em}
         \usecounter{problemnum}}}{\end{list}}
\makeatletter
\newcommand{\problem}[1][{}]{\item \let\@currentlabel=\theproblem \textbf{#1}}
\makeatother

\newcounter{problempartnum}[problemnum]
\newenvironment{problemparts}{
        \begin{list}{{\bf (\alph{problempartnum})}}
        {\setlength{\leftmargin}{2.5em}
         \setlength{\rightmargin}{2.5em}
         \setlength{\labelsep}{0.5em}}}{\end{list}}
\newcommand{\problempart}{\addtocounter{problempartnum}{1}\item} 
 
\newcommand{\hint}{{\em Hint:\ }}

\newcommand{\points}[1]{[#1 points]\ }
\newcommand{\parts}[1]
{
  \ifnum#1=1
  (1 part)
  \else
  (#1 parts)
  \fi
  \ 
}

\newcommand{\defn}[1]{{\boldmath\textit{\textbf{#1}}}}
\newcommand{\defi}[1]{{\textit{\textbf{#1\/}}}}
 
\setlength{\oddsidemargin}{0pt}
\setlength{\evensidemargin}{0pt}
\setlength{\textwidth}{6.5in}
\setlength{\topmargin}{0in}
\setlength{\textheight}{8.5in}



% Fill these in!
\newcommand{\theproblemsetnum}{6}
\renewcommand{\theproblemnum}{2}
\newcommand{\releasedate}{November 22, 2013}
\newcommand{\tabUnit}{3ex}
\newcommand{\tab}{\hspace*{\tabUnit}}

\newcommand{\yourname}{YOUR NAME HERE}  %TODO
\newcommand{\yourcollaborators}{YOUR COLLABORATORS HERE} %TODO
\newcommand{\yourrecitation}{YOUR RECITATION HERE} %TODO

\begin{document}

\begin{center}
\handout{\# 6}{Problem Set \theproblemsetnum, Problem \theproblemnum}{\releasedate}
\begin{large} \yourname \end{large} \\
\begin{large} Collaborators:  \yourcollaborators \end{large} \\
\begin{large} Recitation: \yourrecitation \end{large}
\end{center}

\hrulefill
\medskip

This problem set is due {\bf Thursday, December 5} at {\bf 11:59PM}. \\

This solution template should be turned in through \color{blue} \href{https://alg.csail.mit.edu}{our submission site}. \color{black}   \footnote{Register an account, if you haven't done so.  Then go to Homework, Problem Set 6, and upload your files.  } \\

For written questions, full credit will be given only to
correct solutions that are described clearly {\em and concisely}. \\

Please fill in the TA and recitation section you attend. Otherwise you may not be able to get your problem sets back in section! \\

\medskip

\hrulefill

\newpage

\pagestyle{myheadings}


\begin{problems}

\problem \points{35} \textbf{Adaptive Algorithms: Fast If Lucky}

Your goal in this problem is to develop an algorithm to compute the
\defn{edit distance} $e(x,y)$ between two $n$-character strings $x$ and $y$,
that is, the minimum number of character insertions, deletions, and/or
substitutions to transform $x$ into~$y$.
Unlike the $O(n^2)$ algorithm presented in class, however, your algorithm will
run faster whenever $e(x,y)$ happens to be $o(n)$.

\begin{problemparts}

\problempart \points{10}
Suppose you knew an upper bound $E$ such that $e(x,y) \le E$.
Describe how to compute $e(x,y)$ in $O(n \cdot E)$ time.
Be sure to analyze your algorithm..

\fbox{\vbox{
%TODO
Write something here!
}}

\problempart \points{10}
Given a value $E$, how would you determine whether $e(x,y) \le E$
in $O(n \cdot E)$ time?

\hint Use your algorithm from part (a).  How can you tell whether it succeeded?

\fbox{\vbox{
%TODO
Write something here!
}}

\problempart \points{15}
Now give an algorithm to compute $e(x,y)$ in $O(n \cdot e(x,y))$ time
without any assumptions.  (In particular, the to-be-computed value of $e(x,y)$
is not known to your algorithm ahead of time.)

\hint Use your algorithm from part (b), for some sequence of choices for~$E$.

\fbox{\vbox{
%TODO
Write something here!
}}

\end{problemparts}
\end{problems}

\end{document}
